\section{Билинейные и квадратичные формы}
Далее рассматриваются билинейные и квадратичные формы на вещественном евклидовом
пространстве $H$.

\begin{definition}
    Пусть $e_1, \dotsc, e_n$ --- базис в $H$, $\phi$ --- билинейная форма.
    Матрица $(a_{ij}) \in \matr{}$, где $a_{ij} = \phi(e_i, e_j), \; i, j =
    \overline{1, n}$ называется \emph{матрицей билинейной формы}.
\end{definition}

\begin{theorem}\label{th:bilinearformop}
    Для каждой \,билинейной формы \, $\phi$ найдётся такой оператор $A \in L(H)$, что
    \begin{equation}\label{eq:bilinearformop}
        \forall x, y \in H \;\; \phi(x, y) = \langle Ax, y\rangle 
    \end{equation}
\end{theorem}

\begin{proof}
    Пусть $e_1, \dotsc, e_n$ --- ортонормированный базис в $H$, $\mathcal{A} =
    (a_{ij}) = (\phi(e_i, e_j))$ --- матрица билинейной формы.

    Рассмотрим оператор, задаваемый матрицей $\mathcal{A}^\top = (a_{ji})$. Он
    задаёт некоторую билинейную форму $f: H^2 \to \fieldr$:
    \[ f(x, y) = \langle Ax, y\rangle \]

    Покажем, что $\phi = f$. 
    \[ f(e_i, e_j) = \langle Ae_i, e_j\rangle = a_{ij} = \phi(e_i, e_j) \]
    $\big(\langle Ae_i, e_j\rangle\big)$ --- транспонированная матрица оператора $A$
\end{proof}

\begin{corollaryth}
    Каждая билинейная форма в $\fieldr^n$ имеет вид
    \[ \phi(x,y) = \sum_{i,j = 1}^n a_{ij} x_i y_j \]
\end{corollaryth}

\begin{corollaryth}
    Билинейная форма \eqref{eq:bilinearformop} симметрична тогда и только тогда,
    когда $A$ самосопряжен.
\end{corollaryth}

\begin{definition}
    Функция $f: H \to \fieldr$ называется \emph{квадратичной формой}, если она может
    быть представлена в виде
    \[ f(x) = \phi(x, x), \]
    где $\phi$ --- билинейная форма.
\end{definition}

Заданной квадратичной форме $f$ может соответствовать бесконечно много
билинейных форм $\phi: H^2 \to \fieldr$ со свойством 
\[ f(x) = \phi(x, x) \quad \forall x \in H \]

Однако если $\phi$ --- симметрическая билинейная форма, то из равенства
\[ \phi(x + y, x+y) = \phi(x, x) + 2\phi(x, y) + \phi(y, y) \]
следует, что
\[ \phi(x, y) = \frac{1}{2} (f(x+y) - f(x) - f(y)) \]

Таким образом для данной квадратичной формы $f$ существует единственная
симметрическая билинейная форма $\phi$, для которой $f(x) = \phi(x, x)$. Такая
билинейная форма $\phi$ называется \emph{полярной} к квадратичной форме $f$.

\begin{definition}
    \emph{Матрицей квадратичной формы} называется матрица соответствующей ей полярной
    билинейной формы.
\end{definition}

Из определения полярной билинейной формы и теоремы \ref{th:bilinearformop}
следует

\begin{theorem}
    Каждая квадратичная форма $f$ может быть единственным образом представлена в
    виде
    \[ f(x) = \langle Ax, x\rangle, \]
    где $A$ --- самосопряженный оператор из $L(H)$.
\end{theorem}

\begin{definition}
    Квадратичная форма $f: H \to \fieldr$ называется \emph{положительно
    определённой}
    (\emph{положительно полуопределённой}), если для любого ненулевого $x \in H$
    \[ f(x) > 0 \quad (f(x) \geq 0) \]
\end{definition}

Из определений вытекает, что \emph{квадратичная форма положительно определена
тогда и
только тогда, когда соответствующий ей оператор $A$ положительно определён.}

\begin{definition}
    Базис $e_1, \dotsc, e_n$ в $H$ называется \emph{каноническим} для квадратичной
    формы $f : H \to \fieldr$, если в этом базисе она имеет вид
    \[ f(x) = \sum_{i=1}^n \lambda_i x_i^2, \quad x = \sum_{j=1}^n x_j e_j, \;
    \lambda_1, \dotsc, \lambda_k \in \fieldr \]

    Коэффициенты $\lambda_k$ называются \emph{каноническими коэффициентами}
    квадратичной формы.
\end{definition}

\begin{theorem}
    Для каждой квадратичной формы существует канонический базис.
\end{theorem}

\begin{proof}
    Квадратичная форма $f : H \to \fieldr$ представима в виде
    \[ f(x) = \langle Ax, x\rangle, \]
    где $A^* = A$. Согласно теореме \ref{th:selfadjointspectrum}, в $H$ существует
    ортонормированный базис, состоящий из собственных векторов оператора $A$.
    \[ Ae_k = \lambda_k e_k, \quad k = \overline{1,n} \]

    Раскладывая вектор $x\in H$ по этому базису получаем:
    \[ f(x) = \langle \sum_{k=1}^n x_k \lambda_k e_k, \sum_{j=1}^n x_j
    e_j\rangle = \sum_{k=1}^n \lambda_k x_k^2 \]
\end{proof}

\begin{remark}
    Пусть $f: H \to \fieldr$ --- квадратичная форма. Используя обозначения из
    доказательства предыдущей теоремы, рассмотрим базис $e_1', \dotsc, e_k'$
    такой, что
    \[ e_k' = 
        \begin{cases}
            e_k, & \lambda_k = 0 \\[0.3em]
            \dfrac{e_k}{\sqrt{\absv{\lambda_k}}}, & \lambda_k \neq 0
    \end{cases} \quad\quad k = \overline{1,n} \]

    Тогда
    \[ f(x) = \langle Ax, x\rangle = \sum_{i=1}^n \lambda_i x_i^2 \norm{e_i}^2
    \]

    Получаем, что
    \[ \lambda_i\norm{e_i}^2 =
        \begin{cases}
            0, & \lambda_i = 0 \\
            \sgn \lambda_i, & \lambda_i \neq 0
    \end{cases} \]
    
    Переставляя элементы базиса, без ограничения общности, можно считать, что
    $\lambda_{m+1} = \dotsb = \lambda_{n} = 0$, $\lambda_1, \dotsc, \lambda_p >
    0$, $\lambda_{p+1}, \dotsc, \lambda_m < 0$. Тогда в выбранном базисе
    квадратичная форма будет иметь вид
    \[ f(x) = x_1^2 + \dotsb + x_p^2 - x_{p+1}^2 - \dotsb - x_m^2 \]

    Такое представление называется \emph{нормальным видом} квадратичной формы.
\end{remark}
