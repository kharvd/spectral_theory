\section{Самосопряженные операторы в вещественных пространствах}
Пусть $H$ --- вещественное евклидово пространство. Расширение пространства $H$
осуществим по схеме, близкой к расширению поля $\fieldr$ до поля $\fieldc$. А
именно, рассмотрим (вещественное) линейное пространство $\widebar{H} = H^2$.
Введём операцию умножения пары $(x, y) \in H^2$ на комплексное число $\alpha +
\beta i$ следующим образом:
\[ (\alpha + \beta i)(x, y) = (\alpha x - \beta y, \beta x + \alpha y) \]

Несложно проверить, что эта операция удовлетворяет требуемым аксиомам умножения
на число в линейном пространстве, а значит пространство $\widebar{H}$ с таким
внешним законом композиции является комплексным.

Каждую пару $(x, y) \in \widebar{H}$, по аналогии с комплексными числами, будем
обозначать $x + i y$.

Определим скалярное произведение в $\widebar{H}$ следующим образом:
\[ \langle x_1 + iy_1, x_2 + iy_2\rangle = \langle x_1, x_2 \rangle + \langle
    y_1, y_2 \rangle + i\langle x_2, y_1\rangle - i\langle x_1, y_2\rangle \]

Теперь $\widebar{H}$ имеет структуру комплексного евклидова пространства.

\begin{definition}
    Построенное пространство $\widebar{H}$ называется
    \emph{комплексификацией} вещественного евклидова пространства $H$.
\end{definition}

\begin{definition}
    Пусть $A \in L(H)$. Линейный оператор $\widebar{A} \in L(\widebar{H})$,
    определённый формулой
    \[ \widebar{A}(x, y) = (Ax, Ay), \; (x, y) \in \widebar{H}, \]
    или иначе
    \[ \widebar{A}(x + iy) = Ax + iAy, \; x + iy \in \widebar{H}, \]
    называется \emph{расширением} оператора $A$ на $H$.
\end{definition}

\begin{remark}
    Пусть $e_1, \dotsc, e_n$ образуют базис в $H$. Тогда векторы $(e_1, 0), \dotsc, (e_n, 0)$
    образуют базис в $\widebar{H}$ (докажите!)
\end{remark}

\begin{remark}
    Из предыдущего замечания следует, что матрицы операторов $A$ и $\widebar{A}$ в одном и том же
    базисе совпадают.
\end{remark}

\begin{remark}
    Пусть $A \in L(H)$.
    \begin{multline*}
        \langle \widebar{A}(x_1 + iy_1), x_2 + iy_2 \rangle = \langle
        Ax_1 + iAy_1, x_2 + iy_2 \rangle =\\= \langle Ax_1, x_2 \rangle + \langle
        Ay_1, y_2 \rangle + i\langle Ay_1, x_2 \rangle - i\langle
        Ax_1, y_2 \rangle =\\= \langle x_1, A^* x_2\rangle + \langle
        y_1, A^* y_2\rangle + i\langle y_1, A^* x_2 \rangle - i\langle
        x_1, A^* y_2\rangle =\\= \langle x_1 + iy_1, A^* x_2 + i A^*
        y_2\rangle = \langle x_1 + iy_1, \widebar{A^*}(x_2+iy_2)\rangle
    \end{multline*}
    Это означает, что оператор, сопряженный к расширению оператора $A$ есть
    расширение $A^*$. Отсюда следует, что если $A = A^*$, то $\widebar{A} =
    \widebar{A}^*$.
\end{remark}

Пусть $A \in L(H)$ --- самосопряженный оператор, $\widebar{A}$ --- его
расширение на $\widebar{H}$ (также самосопряженный). Их матрицы совпадают, поэтому совпадают
характеристические многочлены, а следовательно и спектры. По теореме
\ref{th:selfadjointspectrum} спектр оператора $\widebar{A}$ состоит только из
вещественных собственных значений. Таким образом спектр самосопряженного
оператора в вещественном евклидовом пространстве непуст.

\begin{theorem}
    Пусть $A \in L(H)$ --- самосопряженный оператор. Тогда $\spectrum{A} \neq
    \varnothing$ и существует ортонормированный базис, составленный из
    собственных векторов этого оператора.
\end{theorem}
